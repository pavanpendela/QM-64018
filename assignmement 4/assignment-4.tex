% Options for packages loaded elsewhere
\PassOptionsToPackage{unicode}{hyperref}
\PassOptionsToPackage{hyphens}{url}
%
\documentclass[
]{article}
\usepackage{amsmath,amssymb}
\usepackage{lmodern}
\usepackage{ifxetex,ifluatex}
\ifnum 0\ifxetex 1\fi\ifluatex 1\fi=0 % if pdftex
  \usepackage[T1]{fontenc}
  \usepackage[utf8]{inputenc}
  \usepackage{textcomp} % provide euro and other symbols
\else % if luatex or xetex
  \usepackage{unicode-math}
  \defaultfontfeatures{Scale=MatchLowercase}
  \defaultfontfeatures[\rmfamily]{Ligatures=TeX,Scale=1}
\fi
% Use upquote if available, for straight quotes in verbatim environments
\IfFileExists{upquote.sty}{\usepackage{upquote}}{}
\IfFileExists{microtype.sty}{% use microtype if available
  \usepackage[]{microtype}
  \UseMicrotypeSet[protrusion]{basicmath} % disable protrusion for tt fonts
}{}
\makeatletter
\@ifundefined{KOMAClassName}{% if non-KOMA class
  \IfFileExists{parskip.sty}{%
    \usepackage{parskip}
  }{% else
    \setlength{\parindent}{0pt}
    \setlength{\parskip}{6pt plus 2pt minus 1pt}}
}{% if KOMA class
  \KOMAoptions{parskip=half}}
\makeatother
\usepackage{xcolor}
\IfFileExists{xurl.sty}{\usepackage{xurl}}{} % add URL line breaks if available
\IfFileExists{bookmark.sty}{\usepackage{bookmark}}{\usepackage{hyperref}}
\hypersetup{
  pdftitle={assignment 4},
  pdfauthor={pavan},
  hidelinks,
  pdfcreator={LaTeX via pandoc}}
\urlstyle{same} % disable monospaced font for URLs
\usepackage[margin=1in]{geometry}
\usepackage{color}
\usepackage{fancyvrb}
\newcommand{\VerbBar}{|}
\newcommand{\VERB}{\Verb[commandchars=\\\{\}]}
\DefineVerbatimEnvironment{Highlighting}{Verbatim}{commandchars=\\\{\}}
% Add ',fontsize=\small' for more characters per line
\usepackage{framed}
\definecolor{shadecolor}{RGB}{248,248,248}
\newenvironment{Shaded}{\begin{snugshade}}{\end{snugshade}}
\newcommand{\AlertTok}[1]{\textcolor[rgb]{0.94,0.16,0.16}{#1}}
\newcommand{\AnnotationTok}[1]{\textcolor[rgb]{0.56,0.35,0.01}{\textbf{\textit{#1}}}}
\newcommand{\AttributeTok}[1]{\textcolor[rgb]{0.77,0.63,0.00}{#1}}
\newcommand{\BaseNTok}[1]{\textcolor[rgb]{0.00,0.00,0.81}{#1}}
\newcommand{\BuiltInTok}[1]{#1}
\newcommand{\CharTok}[1]{\textcolor[rgb]{0.31,0.60,0.02}{#1}}
\newcommand{\CommentTok}[1]{\textcolor[rgb]{0.56,0.35,0.01}{\textit{#1}}}
\newcommand{\CommentVarTok}[1]{\textcolor[rgb]{0.56,0.35,0.01}{\textbf{\textit{#1}}}}
\newcommand{\ConstantTok}[1]{\textcolor[rgb]{0.00,0.00,0.00}{#1}}
\newcommand{\ControlFlowTok}[1]{\textcolor[rgb]{0.13,0.29,0.53}{\textbf{#1}}}
\newcommand{\DataTypeTok}[1]{\textcolor[rgb]{0.13,0.29,0.53}{#1}}
\newcommand{\DecValTok}[1]{\textcolor[rgb]{0.00,0.00,0.81}{#1}}
\newcommand{\DocumentationTok}[1]{\textcolor[rgb]{0.56,0.35,0.01}{\textbf{\textit{#1}}}}
\newcommand{\ErrorTok}[1]{\textcolor[rgb]{0.64,0.00,0.00}{\textbf{#1}}}
\newcommand{\ExtensionTok}[1]{#1}
\newcommand{\FloatTok}[1]{\textcolor[rgb]{0.00,0.00,0.81}{#1}}
\newcommand{\FunctionTok}[1]{\textcolor[rgb]{0.00,0.00,0.00}{#1}}
\newcommand{\ImportTok}[1]{#1}
\newcommand{\InformationTok}[1]{\textcolor[rgb]{0.56,0.35,0.01}{\textbf{\textit{#1}}}}
\newcommand{\KeywordTok}[1]{\textcolor[rgb]{0.13,0.29,0.53}{\textbf{#1}}}
\newcommand{\NormalTok}[1]{#1}
\newcommand{\OperatorTok}[1]{\textcolor[rgb]{0.81,0.36,0.00}{\textbf{#1}}}
\newcommand{\OtherTok}[1]{\textcolor[rgb]{0.56,0.35,0.01}{#1}}
\newcommand{\PreprocessorTok}[1]{\textcolor[rgb]{0.56,0.35,0.01}{\textit{#1}}}
\newcommand{\RegionMarkerTok}[1]{#1}
\newcommand{\SpecialCharTok}[1]{\textcolor[rgb]{0.00,0.00,0.00}{#1}}
\newcommand{\SpecialStringTok}[1]{\textcolor[rgb]{0.31,0.60,0.02}{#1}}
\newcommand{\StringTok}[1]{\textcolor[rgb]{0.31,0.60,0.02}{#1}}
\newcommand{\VariableTok}[1]{\textcolor[rgb]{0.00,0.00,0.00}{#1}}
\newcommand{\VerbatimStringTok}[1]{\textcolor[rgb]{0.31,0.60,0.02}{#1}}
\newcommand{\WarningTok}[1]{\textcolor[rgb]{0.56,0.35,0.01}{\textbf{\textit{#1}}}}
\usepackage{graphicx}
\makeatletter
\def\maxwidth{\ifdim\Gin@nat@width>\linewidth\linewidth\else\Gin@nat@width\fi}
\def\maxheight{\ifdim\Gin@nat@height>\textheight\textheight\else\Gin@nat@height\fi}
\makeatother
% Scale images if necessary, so that they will not overflow the page
% margins by default, and it is still possible to overwrite the defaults
% using explicit options in \includegraphics[width, height, ...]{}
\setkeys{Gin}{width=\maxwidth,height=\maxheight,keepaspectratio}
% Set default figure placement to htbp
\makeatletter
\def\fps@figure{htbp}
\makeatother
\setlength{\emergencystretch}{3em} % prevent overfull lines
\providecommand{\tightlist}{%
  \setlength{\itemsep}{0pt}\setlength{\parskip}{0pt}}
\setcounter{secnumdepth}{-\maxdimen} % remove section numbering
\ifluatex
  \usepackage{selnolig}  % disable illegal ligatures
\fi

\title{assignment 4}
\author{pavan}
\date{23/10/2021}

\begin{document}
\maketitle

\hypertarget{setting-working-directory-and-loading-packages}{%
\section{setting working directory and loading
packages}\label{setting-working-directory-and-loading-packages}}

\begin{Shaded}
\begin{Highlighting}[]
\FunctionTok{setwd}\NormalTok{(}\StringTok{"C:/Users/pavankumar pendela/Desktop/MSBA/Quantiative management Dr.Wu/assignment 4"}\NormalTok{)}
\FunctionTok{library}\NormalTok{(lpSolve)}
\FunctionTok{library}\NormalTok{(lpSolveAPI)}
\end{Highlighting}
\end{Shaded}

\begin{Shaded}
\begin{Highlighting}[]
\NormalTok{lpec }\OtherTok{\textless{}{-}} \FunctionTok{make.lp}\NormalTok{(}\DecValTok{5}\NormalTok{,}\DecValTok{6}\NormalTok{)}
\FunctionTok{set.objfn}\NormalTok{(lpec, }\FunctionTok{c}\NormalTok{(}\DecValTok{622}\NormalTok{,}\DecValTok{614}\NormalTok{,}\DecValTok{630}\NormalTok{,}\DecValTok{641}\NormalTok{,}\DecValTok{645}\NormalTok{,}\DecValTok{649}\NormalTok{))}
\FunctionTok{lp.control}\NormalTok{(lpec, }\AttributeTok{sense =} \StringTok{"min"}\NormalTok{)}
\end{Highlighting}
\end{Shaded}

\begin{verbatim}
## $anti.degen
## [1] "fixedvars" "stalling" 
## 
## $basis.crash
## [1] "none"
## 
## $bb.depthlimit
## [1] -50
## 
## $bb.floorfirst
## [1] "automatic"
## 
## $bb.rule
## [1] "pseudononint" "greedy"       "dynamic"      "rcostfixing" 
## 
## $break.at.first
## [1] FALSE
## 
## $break.at.value
## [1] -1e+30
## 
## $epsilon
##       epsb       epsd      epsel     epsint epsperturb   epspivot 
##      1e-10      1e-09      1e-12      1e-07      1e-05      2e-07 
## 
## $improve
## [1] "dualfeas" "thetagap"
## 
## $infinite
## [1] 1e+30
## 
## $maxpivot
## [1] 250
## 
## $mip.gap
## absolute relative 
##    1e-11    1e-11 
## 
## $negrange
## [1] -1e+06
## 
## $obj.in.basis
## [1] TRUE
## 
## $pivoting
## [1] "devex"    "adaptive"
## 
## $presolve
## [1] "none"
## 
## $scalelimit
## [1] 5
## 
## $scaling
## [1] "geometric"   "equilibrate" "integers"   
## 
## $sense
## [1] "minimize"
## 
## $simplextype
## [1] "dual"   "primal"
## 
## $timeout
## [1] 0
## 
## $verbose
## [1] "neutral"
\end{verbatim}

\begin{Shaded}
\begin{Highlighting}[]
\CommentTok{\# Production Capacity Constraints:}
\FunctionTok{set.row}\NormalTok{(lpec, }\DecValTok{1}\NormalTok{, }\FunctionTok{c}\NormalTok{(}\DecValTok{1}\NormalTok{,}\DecValTok{1}\NormalTok{,}\DecValTok{1}\NormalTok{), }\AttributeTok{indices =} \FunctionTok{c}\NormalTok{(}\DecValTok{1}\NormalTok{,}\DecValTok{2}\NormalTok{,}\DecValTok{3}\NormalTok{))}
\FunctionTok{set.row}\NormalTok{(lpec, }\DecValTok{2}\NormalTok{, }\FunctionTok{c}\NormalTok{(}\DecValTok{1}\NormalTok{,}\DecValTok{1}\NormalTok{,}\DecValTok{1}\NormalTok{), }\AttributeTok{indices =} \FunctionTok{c}\NormalTok{(}\DecValTok{4}\NormalTok{,}\DecValTok{5}\NormalTok{,}\DecValTok{6}\NormalTok{))}
\CommentTok{\# Warehouse demand Constraints:}
\FunctionTok{set.row}\NormalTok{(lpec, }\DecValTok{3}\NormalTok{, }\FunctionTok{c}\NormalTok{(}\DecValTok{1}\NormalTok{,}\DecValTok{1}\NormalTok{), }\AttributeTok{indices =} \FunctionTok{c}\NormalTok{(}\DecValTok{1}\NormalTok{,}\DecValTok{4}\NormalTok{))}
\FunctionTok{set.row}\NormalTok{(lpec, }\DecValTok{4}\NormalTok{, }\FunctionTok{c}\NormalTok{(}\DecValTok{1}\NormalTok{,}\DecValTok{1}\NormalTok{), }\AttributeTok{indices =} \FunctionTok{c}\NormalTok{(}\DecValTok{2}\NormalTok{,}\DecValTok{5}\NormalTok{))}
\FunctionTok{set.row}\NormalTok{(lpec, }\DecValTok{5}\NormalTok{, }\FunctionTok{c}\NormalTok{(}\DecValTok{1}\NormalTok{,}\DecValTok{1}\NormalTok{), }\AttributeTok{indices =} \FunctionTok{c}\NormalTok{(}\DecValTok{3}\NormalTok{,}\DecValTok{6}\NormalTok{))}
\CommentTok{\# Setting the rhs values}
\NormalTok{rhs }\OtherTok{\textless{}{-}} \FunctionTok{c}\NormalTok{(}\DecValTok{100}\NormalTok{,}\DecValTok{120}\NormalTok{,}\DecValTok{80}\NormalTok{,}\DecValTok{60}\NormalTok{,}\DecValTok{70}\NormalTok{)}
\FunctionTok{set.rhs}\NormalTok{(lpec, rhs)}
\CommentTok{\# Setting the constraint type}
\FunctionTok{set.constr.type}\NormalTok{(lpec, }\FunctionTok{c}\NormalTok{(}\StringTok{"\textless{}="}\NormalTok{,}\StringTok{"\textless{}="}\NormalTok{,}\StringTok{"="}\NormalTok{,}\StringTok{"="}\NormalTok{,}\StringTok{"="}\NormalTok{))}
\end{Highlighting}
\end{Shaded}

\begin{Shaded}
\begin{Highlighting}[]
\CommentTok{\# boundary condiiton for the decision variables}
\FunctionTok{set.bounds}\NormalTok{(lpec, }\AttributeTok{lower =} \FunctionTok{rep}\NormalTok{(}\DecValTok{0}\NormalTok{, }\DecValTok{6}\NormalTok{))}
\CommentTok{\# Set the names of the rows (constraints) and columns (decision variables)}
\NormalTok{lp.rownames }\OtherTok{\textless{}{-}} \FunctionTok{c}\NormalTok{(}\StringTok{"Plant A capacity"}\NormalTok{, }\StringTok{"Plant B capacity"}\NormalTok{, }\StringTok{"warehouse 1 demand"}\NormalTok{, }\StringTok{"warehouse 2 demand"}\NormalTok{, }\StringTok{"Warehouse 3 Demand"}\NormalTok{)}
\NormalTok{lp.colnames }\OtherTok{\textless{}{-}} \FunctionTok{c}\NormalTok{(}\StringTok{"PlantA to warehouse1"}\NormalTok{, }\StringTok{"PlantA to warehouse2"}\NormalTok{, }\StringTok{"PlantA to warehouse3"}\NormalTok{, }\StringTok{"PlantB to warehouse1"}\NormalTok{, }\StringTok{"PlantB to warehouse2"}\NormalTok{, }\StringTok{"Plant B to warehouse3"}\NormalTok{)}
\FunctionTok{dimnames}\NormalTok{(lpec) }\OtherTok{\textless{}{-}} \FunctionTok{list}\NormalTok{(lp.rownames, lp.colnames)}
\end{Highlighting}
\end{Shaded}

\begin{Shaded}
\begin{Highlighting}[]
\CommentTok{\# Return the linear programming object to ensure the values are correct}
\NormalTok{lpec}
\end{Highlighting}
\end{Shaded}

\begin{verbatim}
## Model name: 
##                      PlantA to warehouse1   PlantA to warehouse2   PlantA to warehouse3   PlantB to warehouse1   PlantB to warehouse2  Plant B to warehouse3         
## Minimize                              622                    614                    630                    641                    645                    649         
## Plant A capacity                        1                      1                      1                      0                      0                      0  <=  100
## Plant B capacity                        0                      0                      0                      1                      1                      1  <=  120
## warehouse 1 demand                      1                      0                      0                      1                      0                      0   =   80
## warehouse 2 demand                      0                      1                      0                      0                      1                      0   =   60
## Warehouse 3 Demand                      0                      0                      1                      0                      0                      1   =   70
## Kind                                  Std                    Std                    Std                    Std                    Std                    Std         
## Type                                 Real                   Real                   Real                   Real                   Real                   Real         
## Upper                                 Inf                    Inf                    Inf                    Inf                    Inf                    Inf         
## Lower                                   0                      0                      0                      0                      0                      0
\end{verbatim}

\begin{Shaded}
\begin{Highlighting}[]
\FunctionTok{write.lp}\NormalTok{(lpec, }\AttributeTok{filename =} \StringTok{"assignment 4 lp"}\NormalTok{, }\AttributeTok{type =} \StringTok{"lp"}\NormalTok{)}
\end{Highlighting}
\end{Shaded}

\hypertarget{solve-the-linear-program}{%
\section{Solve the linear program}\label{solve-the-linear-program}}

\begin{Shaded}
\begin{Highlighting}[]
\FunctionTok{solve}\NormalTok{(lpec)}
\end{Highlighting}
\end{Shaded}

\begin{verbatim}
## [1] 0
\end{verbatim}

``0'' is the optimal solution for the problem

\begin{Shaded}
\begin{Highlighting}[]
\FunctionTok{get.objective}\NormalTok{(lpec)}
\end{Highlighting}
\end{Shaded}

\begin{verbatim}
## [1] 132790
\end{verbatim}

1,32,790 is the production cost based on the given values

\begin{Shaded}
\begin{Highlighting}[]
\FunctionTok{get.variables}\NormalTok{(lpec)}
\end{Highlighting}
\end{Shaded}

\begin{verbatim}
## [1]  0 60 40 80  0 30
\end{verbatim}

Plant A Units Shipped to warehouse 1: 0 units Plant A Units Shipped to
warehouse 2: 60 units Plant A Units Shipped to warehouse 3: 40 units
Plant B Units Shipped to warehouse 1: 80 units Plant B Units Shipped to
warehouse 2: 0 units

\end{document}
